\documentclass[11pt]{article}

    \usepackage[breakable]{tcolorbox}
    \usepackage{parskip} % Stop auto-indenting (to mimic markdown behaviour)
    
    \usepackage{iftex}
    \ifPDFTeX
    	\usepackage[T1]{fontenc}
    	\usepackage{mathpazo}
    \else
    	\usepackage{fontspec}
    \fi

    % Basic figure setup, for now with no caption control since it's done
    % automatically by Pandoc (which extracts ![](path) syntax from Markdown).
    \usepackage{graphicx}
    % Maintain compatibility with old templates. Remove in nbconvert 6.0
    \let\Oldincludegraphics\includegraphics
    % Ensure that by default, figures have no caption (until we provide a
    % proper Figure object with a Caption API and a way to capture that
    % in the conversion process - todo).
    \usepackage{caption}
    \DeclareCaptionFormat{nocaption}{}
    \captionsetup{format=nocaption,aboveskip=0pt,belowskip=0pt}

    \usepackage{float}
    \floatplacement{figure}{H} % forces figures to be placed at the correct location
    \usepackage{xcolor} % Allow colors to be defined
    \usepackage{enumerate} % Needed for markdown enumerations to work
    \usepackage{geometry} % Used to adjust the document margins
    \usepackage{amsmath} % Equations
    \usepackage{amssymb} % Equations
    \usepackage{textcomp} % defines textquotesingle
    % Hack from http://tex.stackexchange.com/a/47451/13684:
    \AtBeginDocument{%
        \def\PYZsq{\textquotesingle}% Upright quotes in Pygmentized code
    }
    \usepackage{upquote} % Upright quotes for verbatim code
    \usepackage{eurosym} % defines \euro
    \usepackage[mathletters]{ucs} % Extended unicode (utf-8) support
    \usepackage{fancyvrb} % verbatim replacement that allows latex
    \usepackage{grffile} % extends the file name processing of package graphics 
                         % to support a larger range
    \makeatletter % fix for old versions of grffile with XeLaTeX
    \@ifpackagelater{grffile}{2019/11/01}
    {
      % Do nothing on new versions
    }
    {
      \def\Gread@@xetex#1{%
        \IfFileExists{"\Gin@base".bb}%
        {\Gread@eps{\Gin@base.bb}}%
        {\Gread@@xetex@aux#1}%
      }
    }
    \makeatother
    \usepackage[Export]{adjustbox} % Used to constrain images to a maximum size
    \adjustboxset{max size={0.9\linewidth}{0.9\paperheight}}

    % The hyperref package gives us a pdf with properly built
    % internal navigation ('pdf bookmarks' for the table of contents,
    % internal cross-reference links, web links for URLs, etc.)
    \usepackage{hyperref}
    % The default LaTeX title has an obnoxious amount of whitespace. By default,
    % titling removes some of it. It also provides customization options.
    \usepackage{titling}
    \usepackage{longtable} % longtable support required by pandoc >1.10
    \usepackage{booktabs}  % table support for pandoc > 1.12.2
    \usepackage[inline]{enumitem} % IRkernel/repr support (it uses the enumerate* environment)
    \usepackage[normalem]{ulem} % ulem is needed to support strikethroughs (\sout)
                                % normalem makes italics be italics, not underlines
    \usepackage{mathrsfs}
    

    
    % Colors for the hyperref package
    \definecolor{urlcolor}{rgb}{0,.145,.698}
    \definecolor{linkcolor}{rgb}{.71,0.21,0.01}
    \definecolor{citecolor}{rgb}{.12,.54,.11}

    % ANSI colors
    \definecolor{ansi-black}{HTML}{3E424D}
    \definecolor{ansi-black-intense}{HTML}{282C36}
    \definecolor{ansi-red}{HTML}{E75C58}
    \definecolor{ansi-red-intense}{HTML}{B22B31}
    \definecolor{ansi-green}{HTML}{00A250}
    \definecolor{ansi-green-intense}{HTML}{007427}
    \definecolor{ansi-yellow}{HTML}{DDB62B}
    \definecolor{ansi-yellow-intense}{HTML}{B27D12}
    \definecolor{ansi-blue}{HTML}{208FFB}
    \definecolor{ansi-blue-intense}{HTML}{0065CA}
    \definecolor{ansi-magenta}{HTML}{D160C4}
    \definecolor{ansi-magenta-intense}{HTML}{A03196}
    \definecolor{ansi-cyan}{HTML}{60C6C8}
    \definecolor{ansi-cyan-intense}{HTML}{258F8F}
    \definecolor{ansi-white}{HTML}{C5C1B4}
    \definecolor{ansi-white-intense}{HTML}{A1A6B2}
    \definecolor{ansi-default-inverse-fg}{HTML}{FFFFFF}
    \definecolor{ansi-default-inverse-bg}{HTML}{000000}

    % common color for the border for error outputs.
    \definecolor{outerrorbackground}{HTML}{FFDFDF}

    % commands and environments needed by pandoc snippets
    % extracted from the output of `pandoc -s`
    \providecommand{\tightlist}{%
      \setlength{\itemsep}{0pt}\setlength{\parskip}{0pt}}
    \DefineVerbatimEnvironment{Highlighting}{Verbatim}{commandchars=\\\{\}}
    % Add ',fontsize=\small' for more characters per line
    \newenvironment{Shaded}{}{}
    \newcommand{\KeywordTok}[1]{\textcolor[rgb]{0.00,0.44,0.13}{\textbf{{#1}}}}
    \newcommand{\DataTypeTok}[1]{\textcolor[rgb]{0.56,0.13,0.00}{{#1}}}
    \newcommand{\DecValTok}[1]{\textcolor[rgb]{0.25,0.63,0.44}{{#1}}}
    \newcommand{\BaseNTok}[1]{\textcolor[rgb]{0.25,0.63,0.44}{{#1}}}
    \newcommand{\FloatTok}[1]{\textcolor[rgb]{0.25,0.63,0.44}{{#1}}}
    \newcommand{\CharTok}[1]{\textcolor[rgb]{0.25,0.44,0.63}{{#1}}}
    \newcommand{\StringTok}[1]{\textcolor[rgb]{0.25,0.44,0.63}{{#1}}}
    \newcommand{\CommentTok}[1]{\textcolor[rgb]{0.38,0.63,0.69}{\textit{{#1}}}}
    \newcommand{\OtherTok}[1]{\textcolor[rgb]{0.00,0.44,0.13}{{#1}}}
    \newcommand{\AlertTok}[1]{\textcolor[rgb]{1.00,0.00,0.00}{\textbf{{#1}}}}
    \newcommand{\FunctionTok}[1]{\textcolor[rgb]{0.02,0.16,0.49}{{#1}}}
    \newcommand{\RegionMarkerTok}[1]{{#1}}
    \newcommand{\ErrorTok}[1]{\textcolor[rgb]{1.00,0.00,0.00}{\textbf{{#1}}}}
    \newcommand{\NormalTok}[1]{{#1}}
    
    % Additional commands for more recent versions of Pandoc
    \newcommand{\ConstantTok}[1]{\textcolor[rgb]{0.53,0.00,0.00}{{#1}}}
    \newcommand{\SpecialCharTok}[1]{\textcolor[rgb]{0.25,0.44,0.63}{{#1}}}
    \newcommand{\VerbatimStringTok}[1]{\textcolor[rgb]{0.25,0.44,0.63}{{#1}}}
    \newcommand{\SpecialStringTok}[1]{\textcolor[rgb]{0.73,0.40,0.53}{{#1}}}
    \newcommand{\ImportTok}[1]{{#1}}
    \newcommand{\DocumentationTok}[1]{\textcolor[rgb]{0.73,0.13,0.13}{\textit{{#1}}}}
    \newcommand{\AnnotationTok}[1]{\textcolor[rgb]{0.38,0.63,0.69}{\textbf{\textit{{#1}}}}}
    \newcommand{\CommentVarTok}[1]{\textcolor[rgb]{0.38,0.63,0.69}{\textbf{\textit{{#1}}}}}
    \newcommand{\VariableTok}[1]{\textcolor[rgb]{0.10,0.09,0.49}{{#1}}}
    \newcommand{\ControlFlowTok}[1]{\textcolor[rgb]{0.00,0.44,0.13}{\textbf{{#1}}}}
    \newcommand{\OperatorTok}[1]{\textcolor[rgb]{0.40,0.40,0.40}{{#1}}}
    \newcommand{\BuiltInTok}[1]{{#1}}
    \newcommand{\ExtensionTok}[1]{{#1}}
    \newcommand{\PreprocessorTok}[1]{\textcolor[rgb]{0.74,0.48,0.00}{{#1}}}
    \newcommand{\AttributeTok}[1]{\textcolor[rgb]{0.49,0.56,0.16}{{#1}}}
    \newcommand{\InformationTok}[1]{\textcolor[rgb]{0.38,0.63,0.69}{\textbf{\textit{{#1}}}}}
    \newcommand{\WarningTok}[1]{\textcolor[rgb]{0.38,0.63,0.69}{\textbf{\textit{{#1}}}}}
    
    
    % Define a nice break command that doesn't care if a line doesn't already
    % exist.
    \def\br{\hspace*{\fill} \\* }
    % Math Jax compatibility definitions
    \def\gt{>}
    \def\lt{<}
    \let\Oldtex\TeX
    \let\Oldlatex\LaTeX
    \renewcommand{\TeX}{\textrm{\Oldtex}}
    \renewcommand{\LaTeX}{\textrm{\Oldlatex}}
    % Document parameters
    % Document title
    \title{1\_fba}
    
    
    
    
    
% Pygments definitions
\makeatletter
\def\PY@reset{\let\PY@it=\relax \let\PY@bf=\relax%
    \let\PY@ul=\relax \let\PY@tc=\relax%
    \let\PY@bc=\relax \let\PY@ff=\relax}
\def\PY@tok#1{\csname PY@tok@#1\endcsname}
\def\PY@toks#1+{\ifx\relax#1\empty\else%
    \PY@tok{#1}\expandafter\PY@toks\fi}
\def\PY@do#1{\PY@bc{\PY@tc{\PY@ul{%
    \PY@it{\PY@bf{\PY@ff{#1}}}}}}}
\def\PY#1#2{\PY@reset\PY@toks#1+\relax+\PY@do{#2}}

\@namedef{PY@tok@w}{\def\PY@tc##1{\textcolor[rgb]{0.73,0.73,0.73}{##1}}}
\@namedef{PY@tok@c}{\let\PY@it=\textit\def\PY@tc##1{\textcolor[rgb]{0.25,0.50,0.50}{##1}}}
\@namedef{PY@tok@cp}{\def\PY@tc##1{\textcolor[rgb]{0.74,0.48,0.00}{##1}}}
\@namedef{PY@tok@k}{\let\PY@bf=\textbf\def\PY@tc##1{\textcolor[rgb]{0.00,0.50,0.00}{##1}}}
\@namedef{PY@tok@kp}{\def\PY@tc##1{\textcolor[rgb]{0.00,0.50,0.00}{##1}}}
\@namedef{PY@tok@kt}{\def\PY@tc##1{\textcolor[rgb]{0.69,0.00,0.25}{##1}}}
\@namedef{PY@tok@o}{\def\PY@tc##1{\textcolor[rgb]{0.40,0.40,0.40}{##1}}}
\@namedef{PY@tok@ow}{\let\PY@bf=\textbf\def\PY@tc##1{\textcolor[rgb]{0.67,0.13,1.00}{##1}}}
\@namedef{PY@tok@nb}{\def\PY@tc##1{\textcolor[rgb]{0.00,0.50,0.00}{##1}}}
\@namedef{PY@tok@nf}{\def\PY@tc##1{\textcolor[rgb]{0.00,0.00,1.00}{##1}}}
\@namedef{PY@tok@nc}{\let\PY@bf=\textbf\def\PY@tc##1{\textcolor[rgb]{0.00,0.00,1.00}{##1}}}
\@namedef{PY@tok@nn}{\let\PY@bf=\textbf\def\PY@tc##1{\textcolor[rgb]{0.00,0.00,1.00}{##1}}}
\@namedef{PY@tok@ne}{\let\PY@bf=\textbf\def\PY@tc##1{\textcolor[rgb]{0.82,0.25,0.23}{##1}}}
\@namedef{PY@tok@nv}{\def\PY@tc##1{\textcolor[rgb]{0.10,0.09,0.49}{##1}}}
\@namedef{PY@tok@no}{\def\PY@tc##1{\textcolor[rgb]{0.53,0.00,0.00}{##1}}}
\@namedef{PY@tok@nl}{\def\PY@tc##1{\textcolor[rgb]{0.63,0.63,0.00}{##1}}}
\@namedef{PY@tok@ni}{\let\PY@bf=\textbf\def\PY@tc##1{\textcolor[rgb]{0.60,0.60,0.60}{##1}}}
\@namedef{PY@tok@na}{\def\PY@tc##1{\textcolor[rgb]{0.49,0.56,0.16}{##1}}}
\@namedef{PY@tok@nt}{\let\PY@bf=\textbf\def\PY@tc##1{\textcolor[rgb]{0.00,0.50,0.00}{##1}}}
\@namedef{PY@tok@nd}{\def\PY@tc##1{\textcolor[rgb]{0.67,0.13,1.00}{##1}}}
\@namedef{PY@tok@s}{\def\PY@tc##1{\textcolor[rgb]{0.73,0.13,0.13}{##1}}}
\@namedef{PY@tok@sd}{\let\PY@it=\textit\def\PY@tc##1{\textcolor[rgb]{0.73,0.13,0.13}{##1}}}
\@namedef{PY@tok@si}{\let\PY@bf=\textbf\def\PY@tc##1{\textcolor[rgb]{0.73,0.40,0.53}{##1}}}
\@namedef{PY@tok@se}{\let\PY@bf=\textbf\def\PY@tc##1{\textcolor[rgb]{0.73,0.40,0.13}{##1}}}
\@namedef{PY@tok@sr}{\def\PY@tc##1{\textcolor[rgb]{0.73,0.40,0.53}{##1}}}
\@namedef{PY@tok@ss}{\def\PY@tc##1{\textcolor[rgb]{0.10,0.09,0.49}{##1}}}
\@namedef{PY@tok@sx}{\def\PY@tc##1{\textcolor[rgb]{0.00,0.50,0.00}{##1}}}
\@namedef{PY@tok@m}{\def\PY@tc##1{\textcolor[rgb]{0.40,0.40,0.40}{##1}}}
\@namedef{PY@tok@gh}{\let\PY@bf=\textbf\def\PY@tc##1{\textcolor[rgb]{0.00,0.00,0.50}{##1}}}
\@namedef{PY@tok@gu}{\let\PY@bf=\textbf\def\PY@tc##1{\textcolor[rgb]{0.50,0.00,0.50}{##1}}}
\@namedef{PY@tok@gd}{\def\PY@tc##1{\textcolor[rgb]{0.63,0.00,0.00}{##1}}}
\@namedef{PY@tok@gi}{\def\PY@tc##1{\textcolor[rgb]{0.00,0.63,0.00}{##1}}}
\@namedef{PY@tok@gr}{\def\PY@tc##1{\textcolor[rgb]{1.00,0.00,0.00}{##1}}}
\@namedef{PY@tok@ge}{\let\PY@it=\textit}
\@namedef{PY@tok@gs}{\let\PY@bf=\textbf}
\@namedef{PY@tok@gp}{\let\PY@bf=\textbf\def\PY@tc##1{\textcolor[rgb]{0.00,0.00,0.50}{##1}}}
\@namedef{PY@tok@go}{\def\PY@tc##1{\textcolor[rgb]{0.53,0.53,0.53}{##1}}}
\@namedef{PY@tok@gt}{\def\PY@tc##1{\textcolor[rgb]{0.00,0.27,0.87}{##1}}}
\@namedef{PY@tok@err}{\def\PY@bc##1{{\setlength{\fboxsep}{\string -\fboxrule}\fcolorbox[rgb]{1.00,0.00,0.00}{1,1,1}{\strut ##1}}}}
\@namedef{PY@tok@kc}{\let\PY@bf=\textbf\def\PY@tc##1{\textcolor[rgb]{0.00,0.50,0.00}{##1}}}
\@namedef{PY@tok@kd}{\let\PY@bf=\textbf\def\PY@tc##1{\textcolor[rgb]{0.00,0.50,0.00}{##1}}}
\@namedef{PY@tok@kn}{\let\PY@bf=\textbf\def\PY@tc##1{\textcolor[rgb]{0.00,0.50,0.00}{##1}}}
\@namedef{PY@tok@kr}{\let\PY@bf=\textbf\def\PY@tc##1{\textcolor[rgb]{0.00,0.50,0.00}{##1}}}
\@namedef{PY@tok@bp}{\def\PY@tc##1{\textcolor[rgb]{0.00,0.50,0.00}{##1}}}
\@namedef{PY@tok@fm}{\def\PY@tc##1{\textcolor[rgb]{0.00,0.00,1.00}{##1}}}
\@namedef{PY@tok@vc}{\def\PY@tc##1{\textcolor[rgb]{0.10,0.09,0.49}{##1}}}
\@namedef{PY@tok@vg}{\def\PY@tc##1{\textcolor[rgb]{0.10,0.09,0.49}{##1}}}
\@namedef{PY@tok@vi}{\def\PY@tc##1{\textcolor[rgb]{0.10,0.09,0.49}{##1}}}
\@namedef{PY@tok@vm}{\def\PY@tc##1{\textcolor[rgb]{0.10,0.09,0.49}{##1}}}
\@namedef{PY@tok@sa}{\def\PY@tc##1{\textcolor[rgb]{0.73,0.13,0.13}{##1}}}
\@namedef{PY@tok@sb}{\def\PY@tc##1{\textcolor[rgb]{0.73,0.13,0.13}{##1}}}
\@namedef{PY@tok@sc}{\def\PY@tc##1{\textcolor[rgb]{0.73,0.13,0.13}{##1}}}
\@namedef{PY@tok@dl}{\def\PY@tc##1{\textcolor[rgb]{0.73,0.13,0.13}{##1}}}
\@namedef{PY@tok@s2}{\def\PY@tc##1{\textcolor[rgb]{0.73,0.13,0.13}{##1}}}
\@namedef{PY@tok@sh}{\def\PY@tc##1{\textcolor[rgb]{0.73,0.13,0.13}{##1}}}
\@namedef{PY@tok@s1}{\def\PY@tc##1{\textcolor[rgb]{0.73,0.13,0.13}{##1}}}
\@namedef{PY@tok@mb}{\def\PY@tc##1{\textcolor[rgb]{0.40,0.40,0.40}{##1}}}
\@namedef{PY@tok@mf}{\def\PY@tc##1{\textcolor[rgb]{0.40,0.40,0.40}{##1}}}
\@namedef{PY@tok@mh}{\def\PY@tc##1{\textcolor[rgb]{0.40,0.40,0.40}{##1}}}
\@namedef{PY@tok@mi}{\def\PY@tc##1{\textcolor[rgb]{0.40,0.40,0.40}{##1}}}
\@namedef{PY@tok@il}{\def\PY@tc##1{\textcolor[rgb]{0.40,0.40,0.40}{##1}}}
\@namedef{PY@tok@mo}{\def\PY@tc##1{\textcolor[rgb]{0.40,0.40,0.40}{##1}}}
\@namedef{PY@tok@ch}{\let\PY@it=\textit\def\PY@tc##1{\textcolor[rgb]{0.25,0.50,0.50}{##1}}}
\@namedef{PY@tok@cm}{\let\PY@it=\textit\def\PY@tc##1{\textcolor[rgb]{0.25,0.50,0.50}{##1}}}
\@namedef{PY@tok@cpf}{\let\PY@it=\textit\def\PY@tc##1{\textcolor[rgb]{0.25,0.50,0.50}{##1}}}
\@namedef{PY@tok@c1}{\let\PY@it=\textit\def\PY@tc##1{\textcolor[rgb]{0.25,0.50,0.50}{##1}}}
\@namedef{PY@tok@cs}{\let\PY@it=\textit\def\PY@tc##1{\textcolor[rgb]{0.25,0.50,0.50}{##1}}}

\def\PYZbs{\char`\\}
\def\PYZus{\char`\_}
\def\PYZob{\char`\{}
\def\PYZcb{\char`\}}
\def\PYZca{\char`\^}
\def\PYZam{\char`\&}
\def\PYZlt{\char`\<}
\def\PYZgt{\char`\>}
\def\PYZsh{\char`\#}
\def\PYZpc{\char`\%}
\def\PYZdl{\char`\$}
\def\PYZhy{\char`\-}
\def\PYZsq{\char`\'}
\def\PYZdq{\char`\"}
\def\PYZti{\char`\~}
% for compatibility with earlier versions
\def\PYZat{@}
\def\PYZlb{[}
\def\PYZrb{]}
\makeatother


    % For linebreaks inside Verbatim environment from package fancyvrb. 
    \makeatletter
        \newbox\Wrappedcontinuationbox 
        \newbox\Wrappedvisiblespacebox 
        \newcommand*\Wrappedvisiblespace {\textcolor{red}{\textvisiblespace}} 
        \newcommand*\Wrappedcontinuationsymbol {\textcolor{red}{\llap{\tiny$\m@th\hookrightarrow$}}} 
        \newcommand*\Wrappedcontinuationindent {3ex } 
        \newcommand*\Wrappedafterbreak {\kern\Wrappedcontinuationindent\copy\Wrappedcontinuationbox} 
        % Take advantage of the already applied Pygments mark-up to insert 
        % potential linebreaks for TeX processing. 
        %        {, <, #, %, $, ' and ": go to next line. 
        %        _, }, ^, &, >, - and ~: stay at end of broken line. 
        % Use of \textquotesingle for straight quote. 
        \newcommand*\Wrappedbreaksatspecials {% 
            \def\PYGZus{\discretionary{\char`\_}{\Wrappedafterbreak}{\char`\_}}% 
            \def\PYGZob{\discretionary{}{\Wrappedafterbreak\char`\{}{\char`\{}}% 
            \def\PYGZcb{\discretionary{\char`\}}{\Wrappedafterbreak}{\char`\}}}% 
            \def\PYGZca{\discretionary{\char`\^}{\Wrappedafterbreak}{\char`\^}}% 
            \def\PYGZam{\discretionary{\char`\&}{\Wrappedafterbreak}{\char`\&}}% 
            \def\PYGZlt{\discretionary{}{\Wrappedafterbreak\char`\<}{\char`\<}}% 
            \def\PYGZgt{\discretionary{\char`\>}{\Wrappedafterbreak}{\char`\>}}% 
            \def\PYGZsh{\discretionary{}{\Wrappedafterbreak\char`\#}{\char`\#}}% 
            \def\PYGZpc{\discretionary{}{\Wrappedafterbreak\char`\%}{\char`\%}}% 
            \def\PYGZdl{\discretionary{}{\Wrappedafterbreak\char`\$}{\char`\$}}% 
            \def\PYGZhy{\discretionary{\char`\-}{\Wrappedafterbreak}{\char`\-}}% 
            \def\PYGZsq{\discretionary{}{\Wrappedafterbreak\textquotesingle}{\textquotesingle}}% 
            \def\PYGZdq{\discretionary{}{\Wrappedafterbreak\char`\"}{\char`\"}}% 
            \def\PYGZti{\discretionary{\char`\~}{\Wrappedafterbreak}{\char`\~}}% 
        } 
        % Some characters . , ; ? ! / are not pygmentized. 
        % This macro makes them "active" and they will insert potential linebreaks 
        \newcommand*\Wrappedbreaksatpunct {% 
            \lccode`\~`\.\lowercase{\def~}{\discretionary{\hbox{\char`\.}}{\Wrappedafterbreak}{\hbox{\char`\.}}}% 
            \lccode`\~`\,\lowercase{\def~}{\discretionary{\hbox{\char`\,}}{\Wrappedafterbreak}{\hbox{\char`\,}}}% 
            \lccode`\~`\;\lowercase{\def~}{\discretionary{\hbox{\char`\;}}{\Wrappedafterbreak}{\hbox{\char`\;}}}% 
            \lccode`\~`\:\lowercase{\def~}{\discretionary{\hbox{\char`\:}}{\Wrappedafterbreak}{\hbox{\char`\:}}}% 
            \lccode`\~`\?\lowercase{\def~}{\discretionary{\hbox{\char`\?}}{\Wrappedafterbreak}{\hbox{\char`\?}}}% 
            \lccode`\~`\!\lowercase{\def~}{\discretionary{\hbox{\char`\!}}{\Wrappedafterbreak}{\hbox{\char`\!}}}% 
            \lccode`\~`\/\lowercase{\def~}{\discretionary{\hbox{\char`\/}}{\Wrappedafterbreak}{\hbox{\char`\/}}}% 
            \catcode`\.\active
            \catcode`\,\active 
            \catcode`\;\active
            \catcode`\:\active
            \catcode`\?\active
            \catcode`\!\active
            \catcode`\/\active 
            \lccode`\~`\~ 	
        }
    \makeatother

    \let\OriginalVerbatim=\Verbatim
    \makeatletter
    \renewcommand{\Verbatim}[1][1]{%
        %\parskip\z@skip
        \sbox\Wrappedcontinuationbox {\Wrappedcontinuationsymbol}%
        \sbox\Wrappedvisiblespacebox {\FV@SetupFont\Wrappedvisiblespace}%
        \def\FancyVerbFormatLine ##1{\hsize\linewidth
            \vtop{\raggedright\hyphenpenalty\z@\exhyphenpenalty\z@
                \doublehyphendemerits\z@\finalhyphendemerits\z@
                \strut ##1\strut}%
        }%
        % If the linebreak is at a space, the latter will be displayed as visible
        % space at end of first line, and a continuation symbol starts next line.
        % Stretch/shrink are however usually zero for typewriter font.
        \def\FV@Space {%
            \nobreak\hskip\z@ plus\fontdimen3\font minus\fontdimen4\font
            \discretionary{\copy\Wrappedvisiblespacebox}{\Wrappedafterbreak}
            {\kern\fontdimen2\font}%
        }%
        
        % Allow breaks at special characters using \PYG... macros.
        \Wrappedbreaksatspecials
        % Breaks at punctuation characters . , ; ? ! and / need catcode=\active 	
        \OriginalVerbatim[#1,codes*=\Wrappedbreaksatpunct]%
    }
    \makeatother

    % Exact colors from NB
    \definecolor{incolor}{HTML}{303F9F}
    \definecolor{outcolor}{HTML}{D84315}
    \definecolor{cellborder}{HTML}{CFCFCF}
    \definecolor{cellbackground}{HTML}{F7F7F7}
    
    % prompt
    \makeatletter
    \newcommand{\boxspacing}{\kern\kvtcb@left@rule\kern\kvtcb@boxsep}
    \makeatother
    \newcommand{\prompt}[4]{
        {\ttfamily\llap{{\color{#2}[#3]:\hspace{3pt}#4}}\vspace{-\baselineskip}}
    }
    

    
    % Prevent overflowing lines due to hard-to-break entities
    \sloppy 
    % Setup hyperref package
    \hypersetup{
      breaklinks=true,  % so long urls are correctly broken across lines
      colorlinks=true,
      urlcolor=urlcolor,
      linkcolor=linkcolor,
      citecolor=citecolor,
      }
    % Slightly bigger margins than the latex defaults
    
    \geometry{verbose,tmargin=1in,bmargin=1in,lmargin=1in,rmargin=1in}
    
    

\begin{document}
    
    \maketitle
    
    

    
    \hypertarget{getting-started}{%
\section{1. Getting Started}\label{getting-started}}

    \hypertarget{authors}{%
\subsection{Authors:}\label{authors}}

\begin{itemize}
\tightlist
\item
  Arianna Basile, MRC Toxicology Unit, University of Cambridge
\item
  Francisco Zorrilla, MRC Toxicology Unit, University of Cambridge
\end{itemize}

    \hypertarget{learning-outcomes}{%
\subsection{Learning Outcomes}\label{learning-outcomes}}

In this tutorial you will use
\href{https://cobrapy.readthedocs.io/en/latest/}{cobrapy} to learn the
following:

\begin{itemize}
\tightlist
\item
  \textbf{1.1}: Import a metabolic reconstruction
\item
  \textbf{1.2}: Inspect the reactions of your model
\item
  \textbf{1.3}: Inspect the metabolites in your model
\item
  \textbf{1.4}: Inspect the genes in your model
\item
  \textbf{1.4.1}: Perform in-silico gene knockout experiments
\end{itemize}

    \begin{tcolorbox}[breakable, size=fbox, boxrule=1pt, pad at break*=1mm,colback=cellbackground, colframe=cellborder]
\prompt{In}{incolor}{1}{\boxspacing}
\begin{Verbatim}[commandchars=\\\{\}]
\PY{c+c1}{\PYZsh{} Import required packages}
\PY{k+kn}{import} \PY{n+nn}{cobra}

\PY{c+c1}{\PYZsh{} Enable autocompleting with tab key}
\PY{o}{\PYZpc{}}\PY{k}{config} Completer.use\PYZus{}jedi = False
\end{Verbatim}
\end{tcolorbox}

    \hypertarget{import-a-reconstruction}{%
\subsection{1.1 Import a reconstruction}\label{import-a-reconstruction}}

    The \href{https://sbml.org/}{Systems Biology Markup Language} is an
XML-based standard format for distributing models which has support for
COBRA models through the FBC extension version 2.

Cobrapy has native support for reading and writing SBML with FBCv2.
Please note that all id's in the model must conform to the SBML SID
requirements in order to generate a valid SBML file.

Let's download and import the model of Saccharomyces cerevisiae.

    \begin{tcolorbox}[breakable, size=fbox, boxrule=1pt, pad at break*=1mm,colback=cellbackground, colframe=cellborder]
\prompt{In}{incolor}{2}{\boxspacing}
\begin{Verbatim}[commandchars=\\\{\}]
\PY{n}{model\PYZus{}yeast}\PY{o}{=}\PY{n}{cobra}\PY{o}{.}\PY{n}{io}\PY{o}{.}\PY{n}{read\PYZus{}sbml\PYZus{}model}\PY{p}{(}\PY{l+s+s2}{\PYZdq{}}\PY{l+s+s2}{iMM904.xml.gz}\PY{l+s+s2}{\PYZdq{}}\PY{p}{)}
\end{Verbatim}
\end{tcolorbox}

    The reactions, metabolites, genes, and compartments attributes of the
cobrapy model are a special type of \texttt{list} called a
\texttt{cobra.DictList}, and each one is made up of
\texttt{cobra.Reaction}, \texttt{cobra.Metabolite}, \texttt{cobra.Gene},
\texttt{cobra.Compartment} objects, respectively.

    \begin{tcolorbox}[breakable, size=fbox, boxrule=1pt, pad at break*=1mm,colback=cellbackground, colframe=cellborder]
\prompt{In}{incolor}{3}{\boxspacing}
\begin{Verbatim}[commandchars=\\\{\}]
\PY{n+nb}{print}\PY{p}{(}\PY{l+s+s2}{\PYZdq{}}\PY{l+s+s2}{Reactions: }\PY{l+s+s2}{\PYZdq{}}\PY{p}{,}\PY{n+nb}{len}\PY{p}{(}\PY{n}{model\PYZus{}yeast}\PY{o}{.}\PY{n}{reactions}\PY{p}{)}\PY{p}{)}
\PY{n+nb}{print}\PY{p}{(}\PY{l+s+s2}{\PYZdq{}}\PY{l+s+s2}{Metabolites: }\PY{l+s+s2}{\PYZdq{}}\PY{p}{,}\PY{n+nb}{len}\PY{p}{(}\PY{n}{model\PYZus{}yeast}\PY{o}{.}\PY{n}{metabolites}\PY{p}{)}\PY{p}{)}
\PY{n+nb}{print}\PY{p}{(}\PY{l+s+s2}{\PYZdq{}}\PY{l+s+s2}{Genes: }\PY{l+s+s2}{\PYZdq{}}\PY{p}{,}\PY{n+nb}{len}\PY{p}{(}\PY{n}{model\PYZus{}yeast}\PY{o}{.}\PY{n}{genes}\PY{p}{)}\PY{p}{)}
\PY{n+nb}{print}\PY{p}{(}\PY{l+s+s2}{\PYZdq{}}\PY{l+s+s2}{Compartments: }\PY{l+s+s2}{\PYZdq{}}\PY{p}{,}\PY{n+nb}{len}\PY{p}{(}\PY{n}{model\PYZus{}yeast}\PY{o}{.}\PY{n}{compartments}\PY{p}{)}\PY{p}{)}
\end{Verbatim}
\end{tcolorbox}

    \begin{Verbatim}[commandchars=\\\{\}]
Reactions:  1577
Metabolites:  1226
Genes:  905
Compartments:  8
    \end{Verbatim}

    When using Jupyter notebooks, this type of information is rendered as a
table.

    \begin{tcolorbox}[breakable, size=fbox, boxrule=1pt, pad at break*=1mm,colback=cellbackground, colframe=cellborder]
\prompt{In}{incolor}{4}{\boxspacing}
\begin{Verbatim}[commandchars=\\\{\}]
\PY{n}{model\PYZus{}yeast}
\end{Verbatim}
\end{tcolorbox}

            \begin{tcolorbox}[breakable, size=fbox, boxrule=.5pt, pad at break*=1mm, opacityfill=0]
\prompt{Out}{outcolor}{4}{\boxspacing}
\begin{Verbatim}[commandchars=\\\{\}]
<Model iMM904 at 0x7ff0fba967c0>
\end{Verbatim}
\end{tcolorbox}
        
    Just like a regular list, objects in the \texttt{DictList} can be
retrieved by index. For example, to get the 3rd reaction in the model
(at index 2nd because of 0-indexing):

    \begin{tcolorbox}[breakable, size=fbox, boxrule=1pt, pad at break*=1mm,colback=cellbackground, colframe=cellborder]
\prompt{In}{incolor}{5}{\boxspacing}
\begin{Verbatim}[commandchars=\\\{\}]
\PY{n}{model\PYZus{}yeast}\PY{o}{.}\PY{n}{reactions}\PY{p}{[}\PY{l+m+mi}{2}\PY{p}{]}
\end{Verbatim}
\end{tcolorbox}

            \begin{tcolorbox}[breakable, size=fbox, boxrule=.5pt, pad at break*=1mm, opacityfill=0]
\prompt{Out}{outcolor}{5}{\boxspacing}
\begin{Verbatim}[commandchars=\\\{\}]
<Reaction 13BGHe at 0x7ff10a88fa90>
\end{Verbatim}
\end{tcolorbox}
        
    Additionally, items can be retrieved by their id using the
\texttt{DictList.get\_by\_id()} function. For example, to get the
cytosolic atp metabolite object (the id is \texttt{atp\_c}), we will
inspect metabolites in the section 1.3 For the moment, we can focus on
the reactions of our model.

    \hypertarget{reactions}{%
\subsection{1.2 Reactions}\label{reactions}}

    We will consider the reaction glucose 6-phosphate isomerase, which
interconverts glucose 6-phosphate and fructose 6-phosphate. The reaction
id for this reaction in our test model is PGI. However, if you want to
see the IDs of the first \texttt{N} number of reactions in the
reconstruction, you can run the code below:

    \begin{tcolorbox}[breakable, size=fbox, boxrule=1pt, pad at break*=1mm,colback=cellbackground, colframe=cellborder]
\prompt{In}{incolor}{6}{\boxspacing}
\begin{Verbatim}[commandchars=\\\{\}]
\PY{n}{reaction\PYZus{}ids} \PY{o}{=} \PY{p}{[}\PY{n}{reaction}\PY{o}{.}\PY{n}{id} \PY{k}{for} \PY{n}{reaction} \PY{o+ow}{in} \PY{n}{model\PYZus{}yeast}\PY{o}{.}\PY{n}{reactions}\PY{p}{]}
\PY{n}{N}\PY{o}{=}\PY{l+m+mi}{20}
\PY{n}{reaction\PYZus{}ids}\PY{p}{[}\PY{p}{:}\PY{n}{N}\PY{p}{]}
\end{Verbatim}
\end{tcolorbox}

            \begin{tcolorbox}[breakable, size=fbox, boxrule=.5pt, pad at break*=1mm, opacityfill=0]
\prompt{Out}{outcolor}{6}{\boxspacing}
\begin{Verbatim}[commandchars=\\\{\}]
['CITtcp',
 '13BGH',
 '13BGHe',
 '13GS',
 '16GS',
 '23CAPPD',
 '2DDA7Ptm',
 '2DHPtm',
 '2DOXG6PP',
 '2HBO',
 '2HBt2',
 '2HMHMBQMTm',
 '2HP6MPMOm',
 '2HPMBQMTm',
 '2HPMMBQMOm',
 '2MBACt',
 'EX\_epistest\_SC\_e',
 'EX\_epist\_e',
 '2MBALDt',
 '2MBALDtm']
\end{Verbatim}
\end{tcolorbox}
        
    Now, let's focus on PGI or another reaction of your choice:

    \begin{tcolorbox}[breakable, size=fbox, boxrule=1pt, pad at break*=1mm,colback=cellbackground, colframe=cellborder]
\prompt{In}{incolor}{7}{\boxspacing}
\begin{Verbatim}[commandchars=\\\{\}]
\PY{n}{pgi} \PY{o}{=} \PY{n}{model\PYZus{}yeast}\PY{o}{.}\PY{n}{reactions}\PY{o}{.}\PY{n}{get\PYZus{}by\PYZus{}id}\PY{p}{(}\PY{l+s+s2}{\PYZdq{}}\PY{l+s+s2}{PGI}\PY{l+s+s2}{\PYZdq{}}\PY{p}{)}
\PY{n}{pgi}
\end{Verbatim}
\end{tcolorbox}

            \begin{tcolorbox}[breakable, size=fbox, boxrule=.5pt, pad at break*=1mm, opacityfill=0]
\prompt{Out}{outcolor}{7}{\boxspacing}
\begin{Verbatim}[commandchars=\\\{\}]
<Reaction PGI at 0x7ff10b150fa0>
\end{Verbatim}
\end{tcolorbox}
        
    We can view the full name and reaction catalyzed as strings.

    \begin{tcolorbox}[breakable, size=fbox, boxrule=1pt, pad at break*=1mm,colback=cellbackground, colframe=cellborder]
\prompt{In}{incolor}{8}{\boxspacing}
\begin{Verbatim}[commandchars=\\\{\}]
\PY{n+nb}{print}\PY{p}{(}\PY{n}{pgi}\PY{o}{.}\PY{n}{name}\PY{p}{)}
\PY{n+nb}{print}\PY{p}{(}\PY{n}{pgi}\PY{o}{.}\PY{n}{reaction}\PY{p}{)}
\end{Verbatim}
\end{tcolorbox}

    \begin{Verbatim}[commandchars=\\\{\}]
Glucose-6-phosphate isomerase
g6p\_c <=> f6p\_c
    \end{Verbatim}

    We can also view reaction upper and lower bounds, large numbers,
typically around 1000 or more, are used as infinite limits (unconstained
fluxes). Because the \texttt{pgi.lower\_bound} \textless{} 0, and
\texttt{pgi.upper\_bound} \textgreater{} 0, pgi is reversible.

    \begin{tcolorbox}[breakable, size=fbox, boxrule=1pt, pad at break*=1mm,colback=cellbackground, colframe=cellborder]
\prompt{In}{incolor}{9}{\boxspacing}
\begin{Verbatim}[commandchars=\\\{\}]
\PY{n+nb}{print}\PY{p}{(}\PY{n}{pgi}\PY{o}{.}\PY{n}{lower\PYZus{}bound}\PY{p}{,} \PY{l+s+s2}{\PYZdq{}}\PY{l+s+s2}{\PYZlt{} pgi \PYZlt{}}\PY{l+s+s2}{\PYZdq{}}\PY{p}{,} \PY{n}{pgi}\PY{o}{.}\PY{n}{upper\PYZus{}bound}\PY{p}{)}
\PY{n+nb}{print}\PY{p}{(}\PY{n}{pgi}\PY{o}{.}\PY{n}{reversibility}\PY{p}{)}
\end{Verbatim}
\end{tcolorbox}

    \begin{Verbatim}[commandchars=\\\{\}]
-999999.0 < pgi < 999999.0
True
    \end{Verbatim}

    The lower and upper bound of reactions can also be modified, and the
reversibility attribute will automatically be updated. The preferred
method for manipulating bounds is using reaction.bounds, e.g.

    \begin{tcolorbox}[breakable, size=fbox, boxrule=1pt, pad at break*=1mm,colback=cellbackground, colframe=cellborder]
\prompt{In}{incolor}{10}{\boxspacing}
\begin{Verbatim}[commandchars=\\\{\}]
\PY{c+c1}{\PYZsh{} Save original bounds}
\PY{n}{old\PYZus{}bounds} \PY{o}{=} \PY{n}{pgi}\PY{o}{.}\PY{n}{bounds}

\PY{c+c1}{\PYZsh{} Define and print new bounds}
\PY{n}{pgi}\PY{o}{.}\PY{n}{bounds} \PY{o}{=} \PY{p}{(}\PY{l+m+mi}{0}\PY{p}{,} \PY{l+m+mf}{1000.0}\PY{p}{)}
\PY{n+nb}{print}\PY{p}{(}\PY{l+s+s2}{\PYZdq{}}\PY{l+s+s2}{New bounds: }\PY{l+s+s2}{\PYZdq{}}\PY{p}{,}\PY{n}{pgi}\PY{o}{.}\PY{n}{lower\PYZus{}bound}\PY{p}{,} \PY{l+s+s2}{\PYZdq{}}\PY{l+s+s2}{\PYZlt{} pgi \PYZlt{}}\PY{l+s+s2}{\PYZdq{}}\PY{p}{,} \PY{n}{pgi}\PY{o}{.}\PY{n}{upper\PYZus{}bound}\PY{p}{)}
\PY{n+nb}{print}\PY{p}{(}\PY{l+s+s2}{\PYZdq{}}\PY{l+s+s2}{Reversibility after modification:}\PY{l+s+s2}{\PYZdq{}}\PY{p}{,} \PY{n}{pgi}\PY{o}{.}\PY{n}{reversibility}\PY{p}{)}

\PY{c+c1}{\PYZsh{} Reset bounds and show reversibility}
\PY{n}{pgi}\PY{o}{.}\PY{n}{bounds} \PY{o}{=} \PY{n}{old\PYZus{}bounds}
\PY{n+nb}{print}\PY{p}{(}\PY{l+s+s2}{\PYZdq{}}\PY{l+s+s2}{Reversibility after resetting:}\PY{l+s+s2}{\PYZdq{}}\PY{p}{,} \PY{n}{pgi}\PY{o}{.}\PY{n}{reversibility}\PY{p}{)}
\end{Verbatim}
\end{tcolorbox}

    \begin{Verbatim}[commandchars=\\\{\}]
New bounds:  0 < pgi < 1000.0
Reversibility after modification: False
Reversibility after resetting: True
    \end{Verbatim}

    We can also ensure the reaction is mass balanced. This function will
return elements which violate mass balance. If it comes back empty, then
the reaction is mass balanced.

    \begin{tcolorbox}[breakable, size=fbox, boxrule=1pt, pad at break*=1mm,colback=cellbackground, colframe=cellborder]
\prompt{In}{incolor}{11}{\boxspacing}
\begin{Verbatim}[commandchars=\\\{\}]
\PY{n}{pgi}\PY{o}{.}\PY{n}{check\PYZus{}mass\PYZus{}balance}\PY{p}{(}\PY{p}{)}
\end{Verbatim}
\end{tcolorbox}

            \begin{tcolorbox}[breakable, size=fbox, boxrule=.5pt, pad at break*=1mm, opacityfill=0]
\prompt{Out}{outcolor}{11}{\boxspacing}
\begin{Verbatim}[commandchars=\\\{\}]
\{\}
\end{Verbatim}
\end{tcolorbox}
        
    In order to add a metabolite, we pass in a dictionary with the
metabolite object and its coefficient

    \begin{tcolorbox}[breakable, size=fbox, boxrule=1pt, pad at break*=1mm,colback=cellbackground, colframe=cellborder]
\prompt{In}{incolor}{12}{\boxspacing}
\begin{Verbatim}[commandchars=\\\{\}]
\PY{n}{pgi}\PY{o}{.}\PY{n}{add\PYZus{}metabolites}\PY{p}{(}\PY{p}{\PYZob{}}\PY{n}{model\PYZus{}yeast}\PY{o}{.}\PY{n}{metabolites}\PY{o}{.}\PY{n}{get\PYZus{}by\PYZus{}id}\PY{p}{(}\PY{l+s+s2}{\PYZdq{}}\PY{l+s+s2}{h\PYZus{}c}\PY{l+s+s2}{\PYZdq{}}\PY{p}{)}\PY{p}{:} \PY{o}{\PYZhy{}}\PY{l+m+mi}{1}\PY{p}{\PYZcb{}}\PY{p}{)}
\PY{n}{pgi}\PY{o}{.}\PY{n}{reaction}
\end{Verbatim}
\end{tcolorbox}

            \begin{tcolorbox}[breakable, size=fbox, boxrule=.5pt, pad at break*=1mm, opacityfill=0]
\prompt{Out}{outcolor}{12}{\boxspacing}
\begin{Verbatim}[commandchars=\\\{\}]
'g6p\_c + h\_c <=> f6p\_c'
\end{Verbatim}
\end{tcolorbox}
        
    The reaction is no longer mass balanced

    \begin{tcolorbox}[breakable, size=fbox, boxrule=1pt, pad at break*=1mm,colback=cellbackground, colframe=cellborder]
\prompt{In}{incolor}{13}{\boxspacing}
\begin{Verbatim}[commandchars=\\\{\}]
\PY{n}{pgi}\PY{o}{.}\PY{n}{check\PYZus{}mass\PYZus{}balance}\PY{p}{(}\PY{p}{)}
\end{Verbatim}
\end{tcolorbox}

            \begin{tcolorbox}[breakable, size=fbox, boxrule=.5pt, pad at break*=1mm, opacityfill=0]
\prompt{Out}{outcolor}{13}{\boxspacing}
\begin{Verbatim}[commandchars=\\\{\}]
\{'charge': -1.0, 'H': -1.0\}
\end{Verbatim}
\end{tcolorbox}
        
    We can remove the metabolite, and the reaction will be balanced once
again.

    \begin{tcolorbox}[breakable, size=fbox, boxrule=1pt, pad at break*=1mm,colback=cellbackground, colframe=cellborder]
\prompt{In}{incolor}{14}{\boxspacing}
\begin{Verbatim}[commandchars=\\\{\}]
\PY{n}{pgi}\PY{o}{.}\PY{n}{subtract\PYZus{}metabolites}\PY{p}{(}\PY{p}{\PYZob{}}\PY{n}{model\PYZus{}yeast}\PY{o}{.}\PY{n}{metabolites}\PY{o}{.}\PY{n}{get\PYZus{}by\PYZus{}id}\PY{p}{(}\PY{l+s+s2}{\PYZdq{}}\PY{l+s+s2}{h\PYZus{}c}\PY{l+s+s2}{\PYZdq{}}\PY{p}{)}\PY{p}{:} \PY{o}{\PYZhy{}}\PY{l+m+mi}{1}\PY{p}{\PYZcb{}}\PY{p}{)}
\PY{n+nb}{print}\PY{p}{(}\PY{n}{pgi}\PY{o}{.}\PY{n}{reaction}\PY{p}{)}
\PY{n+nb}{print}\PY{p}{(}\PY{n}{pgi}\PY{o}{.}\PY{n}{check\PYZus{}mass\PYZus{}balance}\PY{p}{(}\PY{p}{)}\PY{p}{)}
\end{Verbatim}
\end{tcolorbox}

    \begin{Verbatim}[commandchars=\\\{\}]
g6p\_c <=> f6p\_c
\{\}
    \end{Verbatim}

    \hypertarget{metabolites}{%
\subsection{1.3 Metabolites}\label{metabolites}}

    We will consider cytosolic atp as our metabolite, which has the id
\texttt{atp\_c} in our test model. However, if you want to see the IDs
of the frist N metabolites in the reconstruction, you can run the code
below:

    \begin{tcolorbox}[breakable, size=fbox, boxrule=1pt, pad at break*=1mm,colback=cellbackground, colframe=cellborder]
\prompt{In}{incolor}{15}{\boxspacing}
\begin{Verbatim}[commandchars=\\\{\}]
\PY{n}{metabolite\PYZus{}ids} \PY{o}{=} \PY{p}{[}\PY{n}{metabolite}\PY{o}{.}\PY{n}{id} \PY{k}{for} \PY{n}{metabolite} \PY{o+ow}{in} \PY{n}{model\PYZus{}yeast}\PY{o}{.}\PY{n}{metabolites}\PY{p}{]}
\PY{n}{N}\PY{o}{=}\PY{l+m+mi}{20}
\PY{n}{metabolite\PYZus{}ids}\PY{p}{[}\PY{p}{:}\PY{n}{N}\PY{p}{]}
\end{Verbatim}
\end{tcolorbox}

            \begin{tcolorbox}[breakable, size=fbox, boxrule=.5pt, pad at break*=1mm, opacityfill=0]
\prompt{Out}{outcolor}{15}{\boxspacing}
\begin{Verbatim}[commandchars=\\\{\}]
['2dr5p\_c',
 '2hb\_c',
 '2hb\_e',
 '2hhxdal\_c',
 '2hp6mbq\_m',
 '2hp6mp\_m',
 '2hpmhmbq\_m',
 '2hpmmbq\_m',
 '2ippm\_c',
 '2kmb\_c',
 '2mahmp\_c',
 '2mbac\_c',
 '2mbac\_e',
 '2mbald\_c',
 '2mbald\_e',
 '2mbald\_m',
 '2mbtoh\_c',
 '2mbtoh\_e',
 '2mbtoh\_m',
 '2mcit\_m']
\end{Verbatim}
\end{tcolorbox}
        
    Now, let's focus on \texttt{atp\_c} or another metabolite of your
choice:

    \begin{tcolorbox}[breakable, size=fbox, boxrule=1pt, pad at break*=1mm,colback=cellbackground, colframe=cellborder]
\prompt{In}{incolor}{16}{\boxspacing}
\begin{Verbatim}[commandchars=\\\{\}]
\PY{n}{atp} \PY{o}{=} \PY{n}{model\PYZus{}yeast}\PY{o}{.}\PY{n}{metabolites}\PY{o}{.}\PY{n}{get\PYZus{}by\PYZus{}id}\PY{p}{(}\PY{l+s+s2}{\PYZdq{}}\PY{l+s+s2}{atp\PYZus{}c}\PY{l+s+s2}{\PYZdq{}}\PY{p}{)}
\PY{n}{atp}
\end{Verbatim}
\end{tcolorbox}

            \begin{tcolorbox}[breakable, size=fbox, boxrule=.5pt, pad at break*=1mm, opacityfill=0]
\prompt{Out}{outcolor}{16}{\boxspacing}
\begin{Verbatim}[commandchars=\\\{\}]
<Metabolite atp\_c at 0x7ff10a4e71c0>
\end{Verbatim}
\end{tcolorbox}
        
    We can print out the metabolite name and compartment (cytosol in this
case) directly as string.

    \begin{tcolorbox}[breakable, size=fbox, boxrule=1pt, pad at break*=1mm,colback=cellbackground, colframe=cellborder]
\prompt{In}{incolor}{17}{\boxspacing}
\begin{Verbatim}[commandchars=\\\{\}]
\PY{n+nb}{print}\PY{p}{(}\PY{n}{atp}\PY{o}{.}\PY{n}{name}\PY{p}{)}
\PY{n+nb}{print}\PY{p}{(}\PY{n}{atp}\PY{o}{.}\PY{n}{compartment}\PY{p}{)}
\end{Verbatim}
\end{tcolorbox}

    \begin{Verbatim}[commandchars=\\\{\}]
ATP C10H12N5O13P3
c
    \end{Verbatim}

    We can see that ATP is a charged molecule in our model.

    \begin{tcolorbox}[breakable, size=fbox, boxrule=1pt, pad at break*=1mm,colback=cellbackground, colframe=cellborder]
\prompt{In}{incolor}{18}{\boxspacing}
\begin{Verbatim}[commandchars=\\\{\}]
\PY{n}{atp}\PY{o}{.}\PY{n}{charge}
\end{Verbatim}
\end{tcolorbox}

            \begin{tcolorbox}[breakable, size=fbox, boxrule=.5pt, pad at break*=1mm, opacityfill=0]
\prompt{Out}{outcolor}{18}{\boxspacing}
\begin{Verbatim}[commandchars=\\\{\}]
-4
\end{Verbatim}
\end{tcolorbox}
        
    We can see the chemical formula for the metabolite as well.

    \begin{tcolorbox}[breakable, size=fbox, boxrule=1pt, pad at break*=1mm,colback=cellbackground, colframe=cellborder]
\prompt{In}{incolor}{19}{\boxspacing}
\begin{Verbatim}[commandchars=\\\{\}]
\PY{n+nb}{print}\PY{p}{(}\PY{n}{atp}\PY{o}{.}\PY{n}{formula}\PY{p}{)}
\end{Verbatim}
\end{tcolorbox}

    \begin{Verbatim}[commandchars=\\\{\}]
C10H12N5O13P3
    \end{Verbatim}

    \hypertarget{genes}{%
\subsection{1.4 Genes}\label{genes}}

    The \texttt{gene\_reaction\_rule} is a boolean representation of the
gene requirements for this reaction to be active as described in
Schellenberger et al 2011 Nature Protocols 6(9):1290-307.

The gene-protein-reaction rules (GPR) are stored as \texttt{GPR\ class}
in the GPR field of a reaction object. A string representation can be
extracted using \texttt{gene\_reaction\_rule} on a Reaction object.

    \begin{tcolorbox}[breakable, size=fbox, boxrule=1pt, pad at break*=1mm,colback=cellbackground, colframe=cellborder]
\prompt{In}{incolor}{20}{\boxspacing}
\begin{Verbatim}[commandchars=\\\{\}]
\PY{n}{gpr\PYZus{}string} \PY{o}{=} \PY{n}{pgi}\PY{o}{.}\PY{n}{gene\PYZus{}reaction\PYZus{}rule}
\PY{n+nb}{print}\PY{p}{(}\PY{n}{gpr\PYZus{}string}\PY{p}{)}
\end{Verbatim}
\end{tcolorbox}

    \begin{Verbatim}[commandchars=\\\{\}]
YBR196C
    \end{Verbatim}

    Corresponding gene objects also exist. These objects are tracked by the
reactions itself, as well as by the model

    \begin{tcolorbox}[breakable, size=fbox, boxrule=1pt, pad at break*=1mm,colback=cellbackground, colframe=cellborder]
\prompt{In}{incolor}{21}{\boxspacing}
\begin{Verbatim}[commandchars=\\\{\}]
\PY{n}{pgi\PYZus{}gene} \PY{o}{=} \PY{n}{model\PYZus{}yeast}\PY{o}{.}\PY{n}{genes}\PY{o}{.}\PY{n}{get\PYZus{}by\PYZus{}id}\PY{p}{(}\PY{l+s+s2}{\PYZdq{}}\PY{l+s+s2}{YBR196C}\PY{l+s+s2}{\PYZdq{}}\PY{p}{)}
\PY{n}{pgi\PYZus{}gene}
\end{Verbatim}
\end{tcolorbox}

            \begin{tcolorbox}[breakable, size=fbox, boxrule=.5pt, pad at break*=1mm, opacityfill=0]
\prompt{Out}{outcolor}{21}{\boxspacing}
\begin{Verbatim}[commandchars=\\\{\}]
<Gene YBR196C at 0x7ff10a7dcf10>
\end{Verbatim}
\end{tcolorbox}
        
    To check that this gene is also on KEGG database to assess the
consistency of the metabolic reconstruction, click here.

    \hypertarget{simulating-knockouts}{%
\subsubsection{1.4.1 Simulating Knockouts}\label{simulating-knockouts}}

    The \texttt{delete\_model\_genes} function will evaluate the GPR and set
the upper and lower bounds to 0 if the reaction is knocked out.

    \begin{tcolorbox}[breakable, size=fbox, boxrule=1pt, pad at break*=1mm,colback=cellbackground, colframe=cellborder]
\prompt{In}{incolor}{22}{\boxspacing}
\begin{Verbatim}[commandchars=\\\{\}]
\PY{n}{model\PYZus{}yeast}\PY{o}{=}\PY{n}{cobra}\PY{o}{.}\PY{n}{io}\PY{o}{.}\PY{n}{read\PYZus{}sbml\PYZus{}model}\PY{p}{(}\PY{l+s+s2}{\PYZdq{}}\PY{l+s+s2}{iMM904.xml.gz}\PY{l+s+s2}{\PYZdq{}}\PY{p}{)}
\PY{n}{pgi}\PY{o}{=}\PY{n}{model\PYZus{}yeast}\PY{o}{.}\PY{n}{reactions}\PY{o}{.}\PY{n}{get\PYZus{}by\PYZus{}id}\PY{p}{(}\PY{l+s+s2}{\PYZdq{}}\PY{l+s+s2}{PGI}\PY{l+s+s2}{\PYZdq{}}\PY{p}{)}
\PY{n+nb}{print}\PY{p}{(}\PY{l+s+s2}{\PYZdq{}}\PY{l+s+s2}{before KO: }\PY{l+s+si}{\PYZpc{}4d}\PY{l+s+s2}{ \PYZlt{} flux\PYZus{}PGI \PYZlt{} }\PY{l+s+si}{\PYZpc{}4d}\PY{l+s+s2}{\PYZdq{}} \PY{o}{\PYZpc{}} \PY{p}{(}\PY{n}{pgi}\PY{o}{.}\PY{n}{lower\PYZus{}bound}\PY{p}{,} \PY{n}{pgi}\PY{o}{.}\PY{n}{upper\PYZus{}bound}\PY{p}{)}\PY{p}{)}


\PY{n}{gene}\PY{o}{=}\PY{n}{model\PYZus{}yeast}\PY{o}{.}\PY{n}{genes}\PY{o}{.}\PY{n}{get\PYZus{}by\PYZus{}id}\PY{p}{(}\PY{l+s+s2}{\PYZdq{}}\PY{l+s+s2}{YBR196C}\PY{l+s+s2}{\PYZdq{}}\PY{p}{)}
\PY{n}{gene}\PY{o}{.}\PY{n}{knock\PYZus{}out}\PY{p}{(}\PY{p}{)}
\PY{n+nb}{print}\PY{p}{(}\PY{l+s+s2}{\PYZdq{}}\PY{l+s+s2}{after KO: }\PY{l+s+si}{\PYZpc{}4d}\PY{l+s+s2}{ \PYZlt{} flux\PYZus{}PGI \PYZlt{} }\PY{l+s+si}{\PYZpc{}4d}\PY{l+s+s2}{\PYZdq{}} \PY{o}{\PYZpc{}} \PY{p}{(}\PY{n}{pgi}\PY{o}{.}\PY{n}{lower\PYZus{}bound}\PY{p}{,} \PY{n}{pgi}\PY{o}{.}\PY{n}{upper\PYZus{}bound}\PY{p}{)}\PY{p}{)}
\end{Verbatim}
\end{tcolorbox}

    \begin{Verbatim}[commandchars=\\\{\}]
before KO: -999999 < flux\_PGI < 999999
after KO:    0 < flux\_PGI <    0
    \end{Verbatim}

    One may often want to make small changes to a model and evaluate their
impacts. For example, we may want to knock-out all reactions
sequentially, and see what the impact of this is on the objective
function. One way of doing this would be to create a new copy of the
model before each knock-out with the \texttt{model.copy()} function.
However, even with small models, this is a very slow approach as models
are quite complex objects. Instead, it is better to carry out the
knock-out, optimize (i.e.~solve FBA problem), and then manually reset
the reaction bounds before proceeding with the next reaction. Since this
is such a common scenario, \texttt{cobrapy} allows us to use the model
as a context, to have changes reverted automatically.

Here we knock out the first N reactions and check for new growth rate
values:

    \begin{tcolorbox}[breakable, size=fbox, boxrule=1pt, pad at break*=1mm,colback=cellbackground, colframe=cellborder]
\prompt{In}{incolor}{23}{\boxspacing}
\begin{Verbatim}[commandchars=\\\{\}]
\PY{c+c1}{\PYZsh{} Import the model again to reverse the previous edits}
\PY{n}{model\PYZus{}yeast}\PY{o}{=}\PY{n}{cobra}\PY{o}{.}\PY{n}{io}\PY{o}{.}\PY{n}{read\PYZus{}sbml\PYZus{}model}\PY{p}{(}\PY{l+s+s2}{\PYZdq{}}\PY{l+s+s2}{iMM904.xml.gz}\PY{l+s+s2}{\PYZdq{}}\PY{p}{)}

\PY{c+c1}{\PYZsh{} Show FBA solution growth rate prior to reaction knockouts}
\PY{n}{model\PYZus{}yeast}\PY{o}{.}\PY{n}{optimize}\PY{p}{(}\PY{p}{)}
\PY{n+nb}{print}\PY{p}{(}\PY{l+s+s2}{\PYZdq{}}\PY{l+s+s2}{Pre reaction knockout growth rate: }\PY{l+s+s2}{\PYZdq{}}\PY{p}{,}\PY{n}{model\PYZus{}yeast}\PY{o}{.}\PY{n}{objective}\PY{o}{.}\PY{n}{value}\PY{p}{)}

\PY{c+c1}{\PYZsh{} Define first N number of reactions to knock out}
\PY{n}{N}\PY{o}{=}\PY{l+m+mi}{20}

\PY{c+c1}{\PYZsh{} Simulate knockouts of N single reactions }
\PY{k}{for} \PY{n}{reaction} \PY{o+ow}{in} \PY{n}{model\PYZus{}yeast}\PY{o}{.}\PY{n}{reactions}\PY{p}{[}\PY{p}{:}\PY{n}{N}\PY{p}{]}\PY{p}{:}
    \PY{k}{with} \PY{n}{model\PYZus{}yeast} \PY{k}{as} \PY{n}{model\PYZus{}yeast}\PY{p}{:}  \PY{c+c1}{\PYZsh{} Prevent editing of the original model}
        \PY{n}{reaction}\PY{o}{.}\PY{n}{knock\PYZus{}out}\PY{p}{(}\PY{p}{)}
        \PY{n}{model\PYZus{}yeast}\PY{o}{.}\PY{n}{optimize}\PY{p}{(}\PY{p}{)}
        \PY{n+nb}{print}\PY{p}{(}\PY{l+s+s1}{\PYZsq{}}\PY{l+s+si}{\PYZpc{}s}\PY{l+s+s1}{ blocked (bounds: }\PY{l+s+si}{\PYZpc{}s}\PY{l+s+s1}{), new growth rate }\PY{l+s+si}{\PYZpc{}f}\PY{l+s+s1}{\PYZsq{}} \PY{o}{\PYZpc{}}
              \PY{p}{(}\PY{n}{reaction}\PY{o}{.}\PY{n}{id}\PY{p}{,} \PY{n+nb}{str}\PY{p}{(}\PY{n}{reaction}\PY{o}{.}\PY{n}{bounds}\PY{p}{)}\PY{p}{,} \PY{n}{model\PYZus{}yeast}\PY{o}{.}\PY{n}{objective}\PY{o}{.}\PY{n}{value}\PY{p}{)}\PY{p}{)}
\end{Verbatim}
\end{tcolorbox}

    \begin{Verbatim}[commandchars=\\\{\}]
Pre reaction knockout growth rate:  0.28786570370401793
CITtcp blocked (bounds: (0, 0)), new growth rate 0.287866
13BGH blocked (bounds: (0, 0)), new growth rate 0.287866
13BGHe blocked (bounds: (0, 0)), new growth rate 0.287866
13GS blocked (bounds: (0, 0)), new growth rate 0.000000
16GS blocked (bounds: (0, 0)), new growth rate 0.287866
23CAPPD blocked (bounds: (0, 0)), new growth rate 0.287866
2DDA7Ptm blocked (bounds: (0, 0)), new growth rate 0.287866
2DHPtm blocked (bounds: (0, 0)), new growth rate 0.287866
2DOXG6PP blocked (bounds: (0, 0)), new growth rate 0.287866
2HBO blocked (bounds: (0, 0)), new growth rate 0.287866
2HBt2 blocked (bounds: (0, 0)), new growth rate 0.287866
2HMHMBQMTm blocked (bounds: (0, 0)), new growth rate 0.287866
2HP6MPMOm blocked (bounds: (0, 0)), new growth rate 0.287866
2HPMBQMTm blocked (bounds: (0, 0)), new growth rate 0.287866
2HPMMBQMOm blocked (bounds: (0, 0)), new growth rate 0.287866
2MBACt blocked (bounds: (0, 0)), new growth rate 0.287866
EX\_epistest\_SC\_e blocked (bounds: (0, 0)), new growth rate 0.287866
EX\_epist\_e blocked (bounds: (0, 0)), new growth rate 0.287866
2MBALDt blocked (bounds: (0, 0)), new growth rate 0.287866
2MBALDtm blocked (bounds: (0, 0)), new growth rate 0.287866
    \end{Verbatim}

    Next we will knock out genes instead of reactions.

    \begin{tcolorbox}[breakable, size=fbox, boxrule=1pt, pad at break*=1mm,colback=cellbackground, colframe=cellborder]
\prompt{In}{incolor}{24}{\boxspacing}
\begin{Verbatim}[commandchars=\\\{\}]
\PY{c+c1}{\PYZsh{} Import the model again to reverse the previous edits}
\PY{n}{model\PYZus{}yeast}\PY{o}{=}\PY{n}{cobra}\PY{o}{.}\PY{n}{io}\PY{o}{.}\PY{n}{read\PYZus{}sbml\PYZus{}model}\PY{p}{(}\PY{l+s+s2}{\PYZdq{}}\PY{l+s+s2}{iMM904.xml.gz}\PY{l+s+s2}{\PYZdq{}}\PY{p}{)}

\PY{c+c1}{\PYZsh{} Show FBA solution growth rate prior to gene knockouts}
\PY{n}{model\PYZus{}yeast}\PY{o}{.}\PY{n}{optimize}\PY{p}{(}\PY{p}{)}
\PY{n+nb}{print}\PY{p}{(}\PY{l+s+s2}{\PYZdq{}}\PY{l+s+s2}{Pre gene knockout growth rate: }\PY{l+s+s2}{\PYZdq{}}\PY{p}{,}\PY{n}{model\PYZus{}yeast}\PY{o}{.}\PY{n}{objective}\PY{o}{.}\PY{n}{value}\PY{p}{)}

\PY{c+c1}{\PYZsh{} Define first N number of genes to knock out}
\PY{n}{N}\PY{o}{=}\PY{l+m+mi}{20}

\PY{c+c1}{\PYZsh{} Simulate knockouts of N genes and print values}
\PY{k}{for} \PY{n}{gene} \PY{o+ow}{in} \PY{n}{model\PYZus{}yeast}\PY{o}{.}\PY{n}{genes}\PY{p}{[}\PY{p}{:}\PY{n}{N}\PY{p}{]}\PY{p}{:}
    \PY{k}{with} \PY{n}{model\PYZus{}yeast} \PY{k}{as} \PY{n}{model\PYZus{}yeast}\PY{p}{:}  \PY{c+c1}{\PYZsh{} Prevent editing the original model}
        \PY{n}{gene}\PY{o}{.}\PY{n}{knock\PYZus{}out}\PY{p}{(}\PY{p}{)}
        \PY{n}{model\PYZus{}yeast}\PY{o}{.}\PY{n}{optimize}\PY{p}{(}\PY{p}{)}
        \PY{n+nb}{print}\PY{p}{(}\PY{l+s+s1}{\PYZsq{}}\PY{l+s+si}{\PYZpc{}s}\PY{l+s+s1}{, new growth rate }\PY{l+s+si}{\PYZpc{}f}\PY{l+s+s1}{\PYZsq{}} \PY{o}{\PYZpc{}}
              \PY{p}{(}\PY{n}{gene}\PY{o}{.}\PY{n}{id}\PY{p}{,} \PY{n}{model\PYZus{}yeast}\PY{o}{.}\PY{n}{objective}\PY{o}{.}\PY{n}{value}\PY{p}{)}\PY{p}{)}
        

\PY{c+c1}{\PYZsh{} Simulate gene knockouts for all genes, this time we are storing the results in a vector for plotting}
\PY{n}{genes\PYZus{}ids}\PY{o}{=}\PY{p}{[}\PY{n}{gene}\PY{o}{.}\PY{n}{id} \PY{k}{for} \PY{n}{gene} \PY{o+ow}{in} \PY{n}{model\PYZus{}yeast}\PY{o}{.}\PY{n}{genes}\PY{p}{]}
\PY{n}{grow\PYZus{}rates}\PY{o}{=}\PY{p}{[}\PY{p}{]}
\PY{k}{for} \PY{n}{gene} \PY{o+ow}{in} \PY{n}{model\PYZus{}yeast}\PY{o}{.}\PY{n}{genes}\PY{p}{:}
    \PY{k}{with} \PY{n}{model\PYZus{}yeast} \PY{k}{as} \PY{n}{model\PYZus{}yeast}\PY{p}{:}  \PY{c+c1}{\PYZsh{} Prevent editing the original model}
        \PY{n}{gene}\PY{o}{.}\PY{n}{knock\PYZus{}out}\PY{p}{(}\PY{p}{)}
        \PY{n}{model\PYZus{}yeast}\PY{o}{.}\PY{n}{optimize}\PY{p}{(}\PY{p}{)}
        \PY{n}{grow\PYZus{}rates}\PY{o}{.}\PY{n}{append}\PY{p}{(}\PY{n}{model\PYZus{}yeast}\PY{o}{.}\PY{n}{objective}\PY{o}{.}\PY{n}{value}\PY{p}{)}
\end{Verbatim}
\end{tcolorbox}

    \begin{Verbatim}[commandchars=\\\{\}]
Pre gene knockout growth rate:  0.28786570370401793
YHR104W, new growth rate 0.287866
YDR368W, new growth rate 0.287866
YGR282C, new growth rate 0.287866
YOL086C, new growth rate 0.287866
YLR300W, new growth rate 0.287866
YFR055W, new growth rate 0.287866
YGL184C, new growth rate 0.287866
YDR261C, new growth rate 0.287866
YDL168W, new growth rate 0.287866
YOR190W, new growth rate 0.287866
YOL030W, new growth rate 0.287866
YLR343W, new growth rate 0.287866
YNL247W, new growth rate 0.287866
YMR303C, new growth rate 0.287866
YGR155W, new growth rate 0.287843
YGR032W, new growth rate 0.287866
YOL132W, new growth rate 0.287866
YGL256W, new growth rate 0.287866
YBR145W, new growth rate 0.287866
YCR034W, new growth rate -0.000000
    \end{Verbatim}

    Let's see the distribution of our results creating a plot with the
vectors obtained above:

    \begin{tcolorbox}[breakable, size=fbox, boxrule=1pt, pad at break*=1mm,colback=cellbackground, colframe=cellborder]
\prompt{In}{incolor}{25}{\boxspacing}
\begin{Verbatim}[commandchars=\\\{\}]
\PY{k+kn}{import} \PY{n+nn}{matplotlib}\PY{n+nn}{.}\PY{n+nn}{pyplot} \PY{k}{as} \PY{n+nn}{plt}
\PY{k+kn}{import} \PY{n+nn}{numpy} \PY{k}{as} \PY{n+nn}{np}

\PY{n}{x} \PY{o}{=} \PY{n}{np}\PY{o}{.}\PY{n}{array}\PY{p}{(}\PY{n}{genes\PYZus{}ids}\PY{p}{)}
\PY{n}{y} \PY{o}{=} \PY{n}{np}\PY{o}{.}\PY{n}{array}\PY{p}{(}\PY{n}{grow\PYZus{}rates}\PY{p}{)}

\PY{n}{plt}\PY{o}{.}\PY{n}{hist}\PY{p}{(}\PY{n}{y}\PY{p}{)}
\PY{n}{plt}\PY{o}{.}\PY{n}{xlabel}\PY{p}{(}\PY{l+s+s1}{\PYZsq{}}\PY{l+s+s1}{Growth rates}\PY{l+s+s1}{\PYZsq{}}\PY{p}{)}
\PY{n}{plt}\PY{o}{.}\PY{n}{ylabel}\PY{p}{(}\PY{l+s+s1}{\PYZsq{}}\PY{l+s+s1}{Number of genes}\PY{l+s+s1}{\PYZsq{}}\PY{p}{)}
\PY{c+c1}{\PYZsh{}plt.show()}
\PY{n}{plt}\PY{o}{.}\PY{n}{savefig}\PY{p}{(}\PY{l+s+s2}{\PYZdq{}}\PY{l+s+s2}{distribution.png}\PY{l+s+s2}{\PYZdq{}}\PY{p}{,} \PY{n}{dpi}\PY{o}{=}\PY{l+m+mi}{100}\PY{p}{,} \PY{n}{bbox\PYZus{}inches}\PY{o}{=}\PY{l+s+s1}{\PYZsq{}}\PY{l+s+s1}{tight}\PY{l+s+s1}{\PYZsq{}}\PY{p}{,}\PY{n}{pad\PYZus{}inches}\PY{o}{=}\PY{l+m+mi}{0}\PY{p}{)}
\end{Verbatim}
\end{tcolorbox}

    \begin{center}
    \adjustimage{max size={0.9\linewidth}{0.9\paperheight}}{output_58_0.png}
    \end{center}
    { \hspace*{\fill} \\}
    
    \hypertarget{questions}{%
\subsubsection{Questions}\label{questions}}

    \begin{enumerate}
\def\labelenumi{\arabic{enumi}.}
\tightlist
\item
  Why does the distribution of predicted growth rates appear to be
  bimodal?
\item
  Can you verify the consistency between gene and reactions knockouts
  results using a gene or a reaction of your choice?
\item
  Can you verify the essentiality of your gene of choice from the
  previous excercise using relevant databases (e.g.~KEGG and the SGD)?
\item
  Do you expect these results to change if we change the medium where we
  are growing our yeast model?
\end{enumerate}

    \hypertarget{solution-to-question-2}{%
\subsubsection{Solution to question 2}\label{solution-to-question-2}}

    \begin{tcolorbox}[breakable, size=fbox, boxrule=1pt, pad at break*=1mm,colback=cellbackground, colframe=cellborder]
\prompt{In}{incolor}{26}{\boxspacing}
\begin{Verbatim}[commandchars=\\\{\}]
\PY{c+c1}{\PYZsh{}type your code here, results in the html}
\end{Verbatim}
\end{tcolorbox}

    \begin{tcolorbox}[breakable, size=fbox, boxrule=1pt, pad at break*=1mm,colback=cellbackground, colframe=cellborder]
\prompt{In}{incolor}{27}{\boxspacing}
\begin{Verbatim}[commandchars=\\\{\}]
\PY{c+c1}{\PYZsh{}type your code here, results in the html}
\end{Verbatim}
\end{tcolorbox}

    \hypertarget{solution-to-question-3}{%
\subsubsection{Solution to question 3}\label{solution-to-question-3}}

When we look for YJL167W in the Saccharomyces genome database
--\textgreater{} HERE , we find out that YJL167W is an essential gene

    \hypertarget{solution-to-question-4}{%
\subsubsection{Solution to question 4}\label{solution-to-question-4}}

Yes, we will look into this phenomenon in the next notebook.


    % Add a bibliography block to the postdoc
    
    
    
\end{document}
